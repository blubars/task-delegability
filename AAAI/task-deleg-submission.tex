\def\year{2018}\relax
%File: formatting-instruction.tex
\documentclass[letterpaper]{article} %DO NOT CHANGE THIS
\usepackage{aaai18}  %Required
\usepackage{times}  %Required
\usepackage{helvet}  %Required
\usepackage{courier}  %Required
\usepackage{url}  %Required
\usepackage{graphicx}  %Required
\frenchspacing  %Required
\setlength{\pdfpagewidth}{8.5in}  %Required
\setlength{\pdfpageheight}{11in}  %Required
%PDF Info Is Required:
  \pdfinfo{
/Title (Not Can We, But Should We: A Design Framework on Appropriate Levels of Automation)
/Author (Chenhao Tan, Brian Lubars)}
\setcounter{secnumdepth}{0}  
 \begin{document}
% The file aaai.sty is the style file for AAAI Press 
% proceedings, working notes, and technical reports.
%
\title{Not Can We, But Should We: \\A Design Framework on Appropriate Levels of Automation}

\author{Chenhao Tan \and Brian Lubars\\
Department of Computer Science\\
University of Colorado, Boulder\\
{chenhao.tan, brian.lubars}@colorado.edu\\
}

%\newcommand{\citet}[1]
%{\citeauthor{#1} ̃\shortcite{#1}}
%\newcommand{\citep}{\cite}
%\newcommand{\citealp}[1]
%{\citeauthor{#1} ̃\citeyear{#1}}

\maketitle
\begin{abstract}
Humans and machines offer complementary abilities. Some tasks are suitable for automation (machine control), some should arguably not be automated (human control), and some are amenable to a mix of the two (human-in-the-loop or machine-in-the-loop designs). What factors go into this classification? When is one paradigm more appropriate than another? By examining factors behind task delegation, human ability, and machine ability for a variety of tasks, we aim to develop a taxonomy of task spaces which help contextualize research and design choices for designing more human-centered automated systems.
\end{abstract}

\section{Task Factors}
To explain the human and task factors behind a task delegation decision, we consider a model with four main dimensions: a person's \textbf{goal} in undertaking the task, their perception of the task's \textbf{difficulty}, their perception of \textbf{risk} associated with accomplishing the task, and finally their \textbf{trust} in the AI agent to accomplish the task.

\subsection{Goal}
The goal dimension examines the reasons someone may be interested in accomplishing a given task and to the expected utility that accompanies the task's successful completion. How important is accomplishing this task, and why? Is the task personally meaningful? Is it a stepping-stone to a larger goal, or an end in itself? 

Goal-setting examines the factors behind an effective goal, especially on the effects of conscious motivation on performance. Factors that make a goal 

We further break down the goal into three sub-factors: the motivation, interest, and utility of the task.
A decision on the delegation of a given task can vary between individuals. A large part of this variability may be explained by differences in the motivation, interest, and utility[cite?]. These factors combined can be summarized as: is the person motivated to actually undertake the task themselves, and does delegating it undermine or support the goal?

\subsection{Difficulty -- needs work on the factors, find papers on social cog theory \& self-efficacy}
Goal-setting theory identifies task difficulty as a primary factor interacting with the goal [cite?]. We propose that an individual's perception of difficulty depends on their self-efficacy and personal experience, their estimates of the cost (time and effort) of accomplishing the goal, and their perception of the task's complexity

A perception of difficulty may be poorly calibrated if the person has little experience with the task. If the person does have experience, we propose that self-efficacy is largely correlated with a perception of the task's difficulty. 

Self-efficacy is defined as [TODO, cite]. It can be thought of as a task-specific self-confidence: a high degree of self-efficacy means that an individual is confident in their ability to accomplish a task. [cite studies as to why we think this might be important]. We propose that self-efficacy can at least partially be explained in terms of personal experience/familiarity with the task and the skills or abilities required. More experience with the task would logically give rise to a deeper understanding of the challenges, strategies, and skills required to complete the task. If the person feels that they have the necessary abilities, then their self-efficacy would rise. [cite] Individuals with low self-efficacy tend to avoid more challenging tasks [cite - "Human Agency in Social Cognitive Theory -- Bandura, 1989]

\subsection{Risk}


\subsection{Trust}
Trust has enjoyed an extensive interest from both within the computer science community and without.

Example cite, statement \cite{lee}.
%\begin{itemize}
%\end{itemize}

%\section{References}
%The references section should be labeled ``References" and should appear at the very end of the paper (don't end the paper with references, and then put a figure by itself on the last page). A sample list of references is given later on in these instructions. Please use a consistent format for references. Poorly prepared or sloppy references reflect badly on the quality of your paper and your research. Please prepare complete and accurate citations.

\bibliography{cite}
\bibliographystyle{aaai}

\end{document}
